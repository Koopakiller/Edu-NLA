%% Generated by Sphinx.
\def\sphinxdocclass{report}
\documentclass[letterpaper,10pt,english]{sphinxmanual}
\ifdefined\pdfpxdimen
   \let\sphinxpxdimen\pdfpxdimen\else\newdimen\sphinxpxdimen
\fi \sphinxpxdimen=.75bp\relax

\usepackage[utf8]{inputenc}
\ifdefined\DeclareUnicodeCharacter
 \ifdefined\DeclareUnicodeCharacterAsOptional
  \DeclareUnicodeCharacter{"00A0}{\nobreakspace}
  \DeclareUnicodeCharacter{"2500}{\sphinxunichar{2500}}
  \DeclareUnicodeCharacter{"2502}{\sphinxunichar{2502}}
  \DeclareUnicodeCharacter{"2514}{\sphinxunichar{2514}}
  \DeclareUnicodeCharacter{"251C}{\sphinxunichar{251C}}
  \DeclareUnicodeCharacter{"2572}{\textbackslash}
 \else
  \DeclareUnicodeCharacter{00A0}{\nobreakspace}
  \DeclareUnicodeCharacter{2500}{\sphinxunichar{2500}}
  \DeclareUnicodeCharacter{2502}{\sphinxunichar{2502}}
  \DeclareUnicodeCharacter{2514}{\sphinxunichar{2514}}
  \DeclareUnicodeCharacter{251C}{\sphinxunichar{251C}}
  \DeclareUnicodeCharacter{2572}{\textbackslash}
 \fi
\fi
\usepackage{cmap}
\usepackage[T1]{fontenc}
\usepackage{amsmath,amssymb,amstext}
\usepackage{babel}
\usepackage{times}
\usepackage[Bjarne]{fncychap}
\usepackage[dontkeepoldnames]{sphinx}

\usepackage{geometry}

% Include hyperref last.
\usepackage{hyperref}
% Fix anchor placement for figures with captions.
\usepackage{hypcap}% it must be loaded after hyperref.
% Set up styles of URL: it should be placed after hyperref.
\urlstyle{same}

\addto\captionsenglish{\renewcommand{\figurename}{Fig.}}
\addto\captionsenglish{\renewcommand{\tablename}{Table}}
\addto\captionsenglish{\renewcommand{\literalblockname}{Listing}}

\addto\captionsenglish{\renewcommand{\literalblockcontinuedname}{continued from previous page}}
\addto\captionsenglish{\renewcommand{\literalblockcontinuesname}{continues on next page}}

\addto\extrasenglish{\def\pageautorefname{page}}

\setcounter{tocdepth}{3}
\setcounter{secnumdepth}{3}


\title{NLA Serie 1 Documentation}
\date{Nov 23, 2017}
\release{1.0}
\author{Tom Lambert, Yuuma Odaka-Falush}
\newcommand{\sphinxlogo}{\vbox{}}
\renewcommand{\releasename}{Release}
\makeindex

\begin{document}

\maketitle
\sphinxtableofcontents
\phantomsection\label{\detokenize{index::doc}}


This is our implementation for Series 1, Numerical Linear Algebra.
The main topic in the problem is object-oriented programming in Python, more specifically working with fractions.
The following documentation explains how our class Fraction and its associated programs work.

A number of automated unit tests were carried out to guarantee the correct execution of the Fraction and Prime classes.
The tests were carried out in the implemented main program.


\chapter{Indices and tables}
\label{\detokenize{index:welcome-to-nla-series-1-s-documentation}}\label{\detokenize{index:indices-and-tables}}\begin{itemize}
\item {} 
\DUrole{xref,std,std-ref}{genindex}

\item {} 
\DUrole{xref,std,std-ref}{modindex}

\item {} 
\DUrole{xref,std,std-ref}{search}

\end{itemize}


\chapter{Modules}
\label{\detokenize{index:modules}}

\section{bruch module}
\label{\detokenize{bruch::doc}}\label{\detokenize{bruch:bruch-module}}

\subsection{Bruch class}
\label{\detokenize{bruch:bruch-class}}
Bruch class exists only to fulfill the task. The actual implementation is in Fraction class.
Bruch inherits from Fraction, thus it has all of the members of Fraction.
See {\hyperref[\detokenize{fraction:fraction-class}]{\sphinxcrossref{\DUrole{std,std-ref}{Members of Fraction class}}}}.
\phantomsection\label{\detokenize{bruch:module-bruch}}\index{bruch (module)}\index{Bruch (class in bruch)}

\begin{fulllineitems}
\phantomsection\label{\detokenize{bruch:bruch.Bruch}}\pysiglinewithargsret{\sphinxbfcode{class }\sphinxcode{bruch.}\sphinxbfcode{Bruch}}{\emph{zaehler}, \emph{nenner}}{}
Bases: {\hyperref[\detokenize{fraction:fraction.Fraction}]{\sphinxcrossref{\sphinxcode{fraction.Fraction}}}}

A derivative of the Fraction-class without other implementations.
For UnitTests see FractionTests.
\index{\_\_init\_\_() (bruch.Bruch method)}

\begin{fulllineitems}
\phantomsection\label{\detokenize{bruch:bruch.Bruch.__init__}}\pysiglinewithargsret{\sphinxbfcode{\_\_init\_\_}}{\emph{zaehler}, \emph{nenner}}{}
Initializes a  new instance.
\begin{quote}\begin{description}
\item[{Parameters}] \leavevmode\begin{itemize}
\item {} 
\sphinxstyleliteralstrong{zaehler} \textendash{} The numerator of the instance to crate.

\item {} 
\sphinxstyleliteralstrong{nenner} \textendash{} The denominator of the instance to create.

\end{itemize}

\end{description}\end{quote}

\end{fulllineitems}


\end{fulllineitems}

\index{main() (in module bruch)}

\begin{fulllineitems}
\phantomsection\label{\detokenize{bruch:bruch.main}}\pysiglinewithargsret{\sphinxcode{bruch.}\sphinxbfcode{main}}{}{}
The main program. It runs unittests to test the main-modules.

\end{fulllineitems}

\index{run\_test() (in module bruch)}

\begin{fulllineitems}
\phantomsection\label{\detokenize{bruch:bruch.run_test}}\pysiglinewithargsret{\sphinxcode{bruch.}\sphinxbfcode{run\_test}}{\emph{class\_name}, \emph{fx}}{}
Runs a given unit-test and prints the result.
\begin{quote}\begin{description}
\item[{Parameters}] \leavevmode\begin{itemize}
\item {} 
\sphinxstyleliteralstrong{class\_name} \textendash{} The class name which will be tested. It will be printed ith the results.

\item {} 
\sphinxstyleliteralstrong{fx} \textendash{} The test-function to execute.

\end{itemize}

\item[{Returns}] \leavevmode
A result-object with the result of the tests.

\end{description}\end{quote}

\end{fulllineitems}



\section{fraction module}
\label{\detokenize{fraction:fraction-module}}\label{\detokenize{fraction::doc}}

\subsection{Mathematical background}
\label{\detokenize{fraction:mathematical-background}}
All calculations obey the rules governing adding, subtracting, multiplying and dividing fractions.
Subtraction and division default to modified addition and multiplication, and the greatest common divisors are always reduced.


\subsection{Members of Fraction class}
\label{\detokenize{fraction:members-of-fraction-class}}\label{\detokenize{fraction:fraction-class}}\label{\detokenize{fraction:module-fraction}}\index{fraction (module)}\index{Fraction (class in fraction)}

\begin{fulllineitems}
\phantomsection\label{\detokenize{fraction:fraction.Fraction}}\pysiglinewithargsret{\sphinxbfcode{class }\sphinxcode{fraction.}\sphinxbfcode{Fraction}}{\emph{numerator}, \emph{denominator}}{}
Represents a fraction with two integers.
\index{\_\_add\_\_() (fraction.Fraction method)}

\begin{fulllineitems}
\phantomsection\label{\detokenize{fraction:fraction.Fraction.__add__}}\pysiglinewithargsret{\sphinxbfcode{\_\_add\_\_}}{\emph{other}}{}
Adds an integer or another fraction to this instance and returns the result.
\begin{quote}\begin{description}
\item[{Parameters}] \leavevmode
\sphinxstyleliteralstrong{other} \textendash{} The other value to add; it can be a Fraction, int or long.

\item[{Returns}] \leavevmode
A new Fraction instance with the result of the addition.

\end{description}\end{quote}

\end{fulllineitems}

\index{\_\_div\_\_() (fraction.Fraction method)}

\begin{fulllineitems}
\phantomsection\label{\detokenize{fraction:fraction.Fraction.__div__}}\pysiglinewithargsret{\sphinxbfcode{\_\_div\_\_}}{\emph{other}}{}
Divides this instance by an integer or another fraction and returns the result.
\begin{quote}\begin{description}
\item[{Parameters}] \leavevmode
\sphinxstyleliteralstrong{other} \textendash{} The other value to divide with; it can be a Fraction, int or long.

\item[{Returns}] \leavevmode
A new Fraction instance with the result of the division.

\end{description}\end{quote}

\end{fulllineitems}

\index{\_\_eq\_\_() (fraction.Fraction method)}

\begin{fulllineitems}
\phantomsection\label{\detokenize{fraction:fraction.Fraction.__eq__}}\pysiglinewithargsret{\sphinxbfcode{\_\_eq\_\_}}{\emph{other}}{}
Compares this Fraction with another Fraction, float, int or long for value-equality.
\begin{quote}\begin{description}
\item[{Parameters}] \leavevmode
\sphinxstyleliteralstrong{other} \textendash{} The other value to compare with; it can be another Fraction, float, int or long

\item[{Returns}] \leavevmode
True, if the values are equal; otherwise False.

\end{description}\end{quote}

\end{fulllineitems}

\index{\_\_float\_\_() (fraction.Fraction method)}

\begin{fulllineitems}
\phantomsection\label{\detokenize{fraction:fraction.Fraction.__float__}}\pysiglinewithargsret{\sphinxbfcode{\_\_float\_\_}}{}{}
Converts the value of this instance into a float-value. The result must not be exact.
\begin{quote}\begin{description}
\item[{Returns}] \leavevmode
A approximated float-value of this instances value.

\end{description}\end{quote}

\end{fulllineitems}

\index{\_\_ge\_\_() (fraction.Fraction method)}

\begin{fulllineitems}
\phantomsection\label{\detokenize{fraction:fraction.Fraction.__ge__}}\pysiglinewithargsret{\sphinxbfcode{\_\_ge\_\_}}{\emph{other}}{}
Checks if the value of this instance is greater or equal than another value.
\begin{quote}\begin{description}
\item[{Parameters}] \leavevmode
\sphinxstyleliteralstrong{other} \textendash{} The other value to compare with; it can be another Fraction float, int or long

\item[{Returns}] \leavevmode
True if this instances value is greater or equal then the other value; otherwise False.

\end{description}\end{quote}

\end{fulllineitems}

\index{\_\_gt\_\_() (fraction.Fraction method)}

\begin{fulllineitems}
\phantomsection\label{\detokenize{fraction:fraction.Fraction.__gt__}}\pysiglinewithargsret{\sphinxbfcode{\_\_gt\_\_}}{\emph{other}}{}
Checks if the value of this instance is greater than another value.
\begin{quote}\begin{description}
\item[{Parameters}] \leavevmode
\sphinxstyleliteralstrong{other} \textendash{} The other value to compare with; it can be another Fraction float, int or long

\item[{Returns}] \leavevmode
True if this instances value is greater then the other value; otherwise False.

\end{description}\end{quote}

\end{fulllineitems}

\index{\_\_init\_\_() (fraction.Fraction method)}

\begin{fulllineitems}
\phantomsection\label{\detokenize{fraction:fraction.Fraction.__init__}}\pysiglinewithargsret{\sphinxbfcode{\_\_init\_\_}}{\emph{numerator}, \emph{denominator}}{}
Initializes a new Fraction-Instance with a value.
\begin{quote}\begin{description}
\item[{Parameters}] \leavevmode\begin{itemize}
\item {} 
\sphinxstyleliteralstrong{numerator} \textendash{} The numerator in the instance to created.

\item {} 
\sphinxstyleliteralstrong{denominator} \textendash{} The denominator in the instance to create.

\end{itemize}

\end{description}\end{quote}

\end{fulllineitems}

\index{\_\_le\_\_() (fraction.Fraction method)}

\begin{fulllineitems}
\phantomsection\label{\detokenize{fraction:fraction.Fraction.__le__}}\pysiglinewithargsret{\sphinxbfcode{\_\_le\_\_}}{\emph{other}}{}
Checks if the value of this instance is less or equal than another value.
\begin{quote}\begin{description}
\item[{Parameters}] \leavevmode
\sphinxstyleliteralstrong{other} \textendash{} The other value to compare with; it can be another Fraction float, int or long

\item[{Returns}] \leavevmode
True if this instances value is less or equal then the other value; otherwise False.

\end{description}\end{quote}

\end{fulllineitems}

\index{\_\_lt\_\_() (fraction.Fraction method)}

\begin{fulllineitems}
\phantomsection\label{\detokenize{fraction:fraction.Fraction.__lt__}}\pysiglinewithargsret{\sphinxbfcode{\_\_lt\_\_}}{\emph{other}}{}
Checks if the value of this instance is less then another value.
\begin{quote}\begin{description}
\item[{Parameters}] \leavevmode
\sphinxstyleliteralstrong{other} \textendash{} The other value to compare with; it can be another Fraction float, int or long

\item[{Returns}] \leavevmode
True if this instances value is less then the other value; otherwise False.

\end{description}\end{quote}

\end{fulllineitems}

\index{\_\_mul\_\_() (fraction.Fraction method)}

\begin{fulllineitems}
\phantomsection\label{\detokenize{fraction:fraction.Fraction.__mul__}}\pysiglinewithargsret{\sphinxbfcode{\_\_mul\_\_}}{\emph{other}}{}
Multiplies an integer or another fraction with this instance and returns the result.
\begin{quote}\begin{description}
\item[{Parameters}] \leavevmode
\sphinxstyleliteralstrong{other} \textendash{} The other value to multiply with; it can be a Fraction, int or long.

\item[{Returns}] \leavevmode
A new Fraction instance with the result of the multiplication.

\end{description}\end{quote}

\end{fulllineitems}

\index{\_\_ne\_\_() (fraction.Fraction method)}

\begin{fulllineitems}
\phantomsection\label{\detokenize{fraction:fraction.Fraction.__ne__}}\pysiglinewithargsret{\sphinxbfcode{\_\_ne\_\_}}{\emph{other}}{}
Compares this Fraction with another Fraction, float, int or long for value-inequality.
\begin{quote}\begin{description}
\item[{Parameters}] \leavevmode
\sphinxstyleliteralstrong{other} \textendash{} The other value to compare with; it can be another Fraction, float, int or long

\item[{Returns}] \leavevmode
False, if the values are equal; otherwise True.

\end{description}\end{quote}

\end{fulllineitems}

\index{\_\_neg\_\_() (fraction.Fraction method)}

\begin{fulllineitems}
\phantomsection\label{\detokenize{fraction:fraction.Fraction.__neg__}}\pysiglinewithargsret{\sphinxbfcode{\_\_neg\_\_}}{}{}
Negates the value of this instance and returns it.
\begin{quote}\begin{description}
\item[{Returns}] \leavevmode
A new Fraction instance with the negated value of this instance

\end{description}\end{quote}

\end{fulllineitems}

\index{\_\_radd\_\_() (fraction.Fraction method)}

\begin{fulllineitems}
\phantomsection\label{\detokenize{fraction:fraction.Fraction.__radd__}}\pysiglinewithargsret{\sphinxbfcode{\_\_radd\_\_}}{\emph{other}}{}
Adds an integer or another fraction to this instance and returns the result.
\begin{quote}\begin{description}
\item[{Parameters}] \leavevmode
\sphinxstyleliteralstrong{other} \textendash{} The other value to add; it can be a Fraction, int or long.

\item[{Returns}] \leavevmode
A new Fraction instance with the result of the addition.

\end{description}\end{quote}

\end{fulllineitems}

\index{\_\_rdiv\_\_() (fraction.Fraction method)}

\begin{fulllineitems}
\phantomsection\label{\detokenize{fraction:fraction.Fraction.__rdiv__}}\pysiglinewithargsret{\sphinxbfcode{\_\_rdiv\_\_}}{\emph{other}}{}
Divides an integer or another fraction with this instance and returns the result.
\begin{quote}\begin{description}
\item[{Parameters}] \leavevmode
\sphinxstyleliteralstrong{other} \textendash{} The other value to divide; it can be a Fraction, int or long.

\item[{Returns}] \leavevmode
A new Fraction instance with the result of the division.

\end{description}\end{quote}

\end{fulllineitems}

\index{\_\_rmul\_\_() (fraction.Fraction method)}

\begin{fulllineitems}
\phantomsection\label{\detokenize{fraction:fraction.Fraction.__rmul__}}\pysiglinewithargsret{\sphinxbfcode{\_\_rmul\_\_}}{\emph{other}}{}
Multiplies an integer or another fraction with this instance and returns the result.
\begin{quote}\begin{description}
\item[{Parameters}] \leavevmode
\sphinxstyleliteralstrong{other} \textendash{} The other value to multiply with; it can be a Fraction, int or long.

\item[{Returns}] \leavevmode
A new Fraction instance with the result of the multiplication.

\end{description}\end{quote}

\end{fulllineitems}

\index{\_\_rsub\_\_() (fraction.Fraction method)}

\begin{fulllineitems}
\phantomsection\label{\detokenize{fraction:fraction.Fraction.__rsub__}}\pysiglinewithargsret{\sphinxbfcode{\_\_rsub\_\_}}{\emph{other}}{}
Subtracts this instance from an integer or another fraction and returns the result.
\begin{quote}\begin{description}
\item[{Parameters}] \leavevmode
\sphinxstyleliteralstrong{other} \textendash{} The other value to subtract from; it can be a Fraction, int or long.

\item[{Returns}] \leavevmode
A new Fraction instance with the result of the subtraction.

\end{description}\end{quote}

\end{fulllineitems}

\index{\_\_str\_\_() (fraction.Fraction method)}

\begin{fulllineitems}
\phantomsection\label{\detokenize{fraction:fraction.Fraction.__str__}}\pysiglinewithargsret{\sphinxbfcode{\_\_str\_\_}}{}{}
Creates a string representation for this instance.
\begin{quote}\begin{description}
\item[{Returns}] \leavevmode
\begin{itemize}
\item {} 
“NaN” if the denominator is 0;

\item {} 
”0” if the denominator is 0;

\item {} 
the numerator-value as a string if the denominator is 1;

\item {} 
otherwise “numerator / denominator”

\end{itemize}


\end{description}\end{quote}

\end{fulllineitems}

\index{\_\_sub\_\_() (fraction.Fraction method)}

\begin{fulllineitems}
\phantomsection\label{\detokenize{fraction:fraction.Fraction.__sub__}}\pysiglinewithargsret{\sphinxbfcode{\_\_sub\_\_}}{\emph{other}}{}
Subtracts an integer or another fraction from this instance and returns the result.
\begin{quote}\begin{description}
\item[{Parameters}] \leavevmode
\sphinxstyleliteralstrong{other} \textendash{} The other value to subtract; it can be a Fraction, int or long.

\item[{Returns}] \leavevmode
A new Fraction instance with the result of the subtraction.

\end{description}\end{quote}

\end{fulllineitems}

\index{clone() (fraction.Fraction method)}

\begin{fulllineitems}
\phantomsection\label{\detokenize{fraction:fraction.Fraction.clone}}\pysiglinewithargsret{\sphinxbfcode{clone}}{}{}
Creates a copy of this instance.
\begin{quote}\begin{description}
\item[{Returns}] \leavevmode
A new instance with the same value as this fraction.

\end{description}\end{quote}

\end{fulllineitems}

\index{reduce() (fraction.Fraction method)}

\begin{fulllineitems}
\phantomsection\label{\detokenize{fraction:fraction.Fraction.reduce}}\pysiglinewithargsret{\sphinxbfcode{reduce}}{}{}
Reduces the fraction by removing all common prime factors.

\end{fulllineitems}


\end{fulllineitems}



\section{prime module}
\label{\detokenize{prime:prime-module}}\label{\detokenize{prime::doc}}

\subsection{Mathematical background}
\label{\detokenize{prime:mathematical-background}}
Prime numbers are a special subset of the naturals.
They can be effectively utilized to find the greatest common divisor of two numbers.


\subsection{Members of Prime class}
\label{\detokenize{prime:module-prime}}\label{\detokenize{prime:members-of-prime-class}}\index{prime (module)}\index{Prime (class in prime)}

\begin{fulllineitems}
\phantomsection\label{\detokenize{prime:prime.Prime}}\pysigline{\sphinxbfcode{class }\sphinxcode{prime.}\sphinxbfcode{Prime}}
Provides methods to obtain prime numbers and use them.
\index{\_\_init\_\_() (prime.Prime method)}

\begin{fulllineitems}
\phantomsection\label{\detokenize{prime:prime.Prime.__init__}}\pysiglinewithargsret{\sphinxbfcode{\_\_init\_\_}}{}{}
This class should not be initialized. All substantial members are static.

\end{fulllineitems}

\index{append\_next\_to\_cache() (prime.Prime static method)}

\begin{fulllineitems}
\phantomsection\label{\detokenize{prime:prime.Prime.append_next_to_cache}}\pysiglinewithargsret{\sphinxbfcode{static }\sphinxbfcode{append\_next\_to\_cache}}{}{}
Calculates the next prime number which is not in the cache.
\begin{quote}\begin{description}
\item[{Returns}] \leavevmode
The added prime number.

\end{description}\end{quote}

\end{fulllineitems}

\index{cache (prime.Prime attribute)}

\begin{fulllineitems}
\phantomsection\label{\detokenize{prime:prime.Prime.cache}}\pysigline{\sphinxbfcode{cache}\sphinxbfcode{ = {[}2, 3, 5, 7{]}}}
\end{fulllineitems}

\index{get\_greatest\_common\_divisor() (prime.Prime static method)}

\begin{fulllineitems}
\phantomsection\label{\detokenize{prime:prime.Prime.get_greatest_common_divisor}}\pysiglinewithargsret{\sphinxbfcode{static }\sphinxbfcode{get\_greatest\_common\_divisor}}{\emph{a}, \emph{b}}{}
Calculates the greatest common divisor
\begin{quote}\begin{description}
\item[{Parameters}] \leavevmode\begin{itemize}
\item {} 
\sphinxstyleliteralstrong{a} \textendash{} The first number.

\item {} 
\sphinxstyleliteralstrong{b} \textendash{} The second number.

\end{itemize}

\item[{Returns}] \leavevmode
The greatest common divisor of a and b.

\end{description}\end{quote}

\end{fulllineitems}

\index{get\_prime() (prime.Prime static method)}

\begin{fulllineitems}
\phantomsection\label{\detokenize{prime:prime.Prime.get_prime}}\pysiglinewithargsret{\sphinxbfcode{static }\sphinxbfcode{get\_prime}}{\emph{index}}{}
Returns the prime number at the given index. The index starts with 0.
\begin{quote}\begin{description}
\item[{Parameters}] \leavevmode
\sphinxstyleliteralstrong{index} (\sphinxstyleliteralemphasis{int}) \textendash{} The index of the requested prime number.

\item[{Returns}] \leavevmode
The prime number at position index.

\end{description}\end{quote}

\end{fulllineitems}

\index{get\_prime\_factors() (prime.Prime static method)}

\begin{fulllineitems}
\phantomsection\label{\detokenize{prime:prime.Prime.get_prime_factors}}\pysiglinewithargsret{\sphinxbfcode{static }\sphinxbfcode{get\_prime\_factors}}{\emph{num}}{}
Returns the prime factors of the given number.
\begin{quote}\begin{description}
\item[{Parameters}] \leavevmode
\sphinxstyleliteralstrong{num} (\sphinxstyleliteralemphasis{long}) \textendash{} The number to split in prime factors.

\item[{Returns}] \leavevmode
An array of prime factors of num.

\item[{Raises}] \leavevmode
\sphinxstyleliteralstrong{ValueError} \textendash{} if num is \textless{}= 1

\end{description}\end{quote}

\end{fulllineitems}


\end{fulllineitems}



\subsection{Remarks}
\label{\detokenize{prime:remarks}}
The generation of prime numbers is accelerated with a cache of already known prime numbers in the RAM.


\renewcommand{\indexname}{Python Module Index}
\begin{sphinxtheindex}
\def\bigletter#1{{\Large\sffamily#1}\nopagebreak\vspace{1mm}}
\bigletter{b}
\item {\sphinxstyleindexentry{bruch}}\sphinxstyleindexpageref{bruch:\detokenize{module-bruch}}
\indexspace
\bigletter{f}
\item {\sphinxstyleindexentry{fraction}}\sphinxstyleindexpageref{fraction:\detokenize{module-fraction}}
\indexspace
\bigletter{p}
\item {\sphinxstyleindexentry{prime}}\sphinxstyleindexpageref{prime:\detokenize{module-prime}}
\end{sphinxtheindex}

\renewcommand{\indexname}{Index}
\printindex
\end{document}